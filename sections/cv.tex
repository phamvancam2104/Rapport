\onecolumn
\section{CV and Global visions of activities}
\subsection{Education}
\noindent
\tb{6-2007}: Baccalaureate of high school Ly Thai To, Bac Ninh, Vietnam.

\noindent
\tb{2007-2012}: Hanoi University of Science and Technology, Vietnam.
\begin{itemize}
	\setlength\itemsep{0.5em}
	\item Programme: Programme de Formation d'Ing\'{e}nieur d'Excellence au Vietnam (PFIEV).
	
	\item Specialty: Information System and Communication.
	
	\item Engineer Degree - Mention: Very well - Ranking: 3rd of the promotion.
\end{itemize}

\noindent
\tb{9/2013-9/2014}: Scholarship student of University of Paris Sud in Master 2 Research.
\begin{itemize}
	\item Specialty: Embedded Systems and Information Processing cooperated by University of Paris Sud, ENS Cachan,
	INSTN and ENSTA ParisTech.
\end{itemize}

\noindent
\tb{From 11/2014}: PhD student at the laboratory of Model-Driven Engineering for Embedded Systems (LISE).


\subsection{Doctoral activities}
\subsubsection{Publications: published, submitted and ongoing}
\begin{itemize}
	\setlength\itemsep{0.5em}
	%\item Interaction components based on ZeroMQ middleware 
	
	\item Rule execution scheduling in incremental model transformation (Modelsward 2016, accepted): as mentioned in \ref{subsec:scheduling}, this work focuses on the execution order of transformation rules in incremental model transformations. 
	Some preliminary results obtained through the experiments conducted with the model transformations defined in the Papyrus Designer framework, which offers a methodology and a framework for component and model-based development for distributed, real-time, embedded systems.
	
	\item Fostering Software Architect and Programmer Collaboration (ASE 2016, submitted): This work is originally motivated by the wish to support better collaboration between software architects and programmers in the project Papyrus-RT. 
	The latter is developed under the cooperation between CEA and ZeligSoft. 
	A generic pattern is proposed to synchronize different artifacts, which are concurrently modified. 
	The pattern is evaluated by experiments, which are detailed in the paper.
	
	\item From UML State Machine to code and back again (SEW 2016, submitted): The objective is to provide a round-trip engineering from UML State Machine to code with some limitations to programmers' manipulation on the code-side. 
	The output is a round-trip engineering approach supporting basic concepts of hierarchical UML State Machine.
	The approach is evaluated by experiments dedicated to different aspects of round-trip engineering and code generation.
	
	\item UML State Machine: Towards a complete code generation solution (to be submitted to ICECCS/APSEC 2016): This work is motivated by the project Papyrus Designer, in which the behavior of components is described using UML state machines. 
	The purpose is to provide a complete code generation solution for full UML State Machine concepts with respect to the UML State Machine Specification, especially the concurrency. 
	Furthermore, the code generated from state machines needs to be efficient in event processing speed and small enough in size to be suitable to micro-controllers. 
	Preliminary experiments show that, for the same state machines, the size of the binary file compiled from the generated code is smaller around 50 times than that produced by the boost framework. 
	Furthermore, the processing speed is faster around 40 times. 
\end{itemize}

\subsubsection{Software Prototypes}
\begin{itemize}
	\setlength\itemsep{0.5em}
	\item CompleteUSM: An Eclipse plug-in, an extension of Papyrus, used to generate full UML State Machine with the support of all modeling features, especially concurrency aspect.
	
	\item PapyrusRTE: An Eclipse plug-in, an extension of Papyrus, used to collaborate software architects and programmers by synchronizing concurrently modified artifacts (model and code), and to help stakeholders produce high quality software.
	
	\item IncRoundtrip: A prototype supporting incremental model transformations in Papyrus Designer.
\end{itemize}

\subsubsection{Non-research activities} ~\\
\tb{Teaching} at IUT

\begin{itemize}
	\setlength\itemsep{0.5em}
	\item Android programming project
	
	\item SQL and database
\end{itemize}

\subsubsection{Visits out of the student team}
\begin{itemize}
	\setlength\itemsep{0.5em}
	\item Summer school DSM-TP: The summer school took place between 23-28 August 2015, in Belgium. This is an international summer school dedicated to practitioners over the world to join and discuss the different prospects of Model-Driven Engineering such as creating a domain specific modeling language or model transformation. 
	
	\item Lise au Vert: This activity took place between 20-21 January, 2016, in Strasbourg, France. Participants are PhD students and post-docs of the laboratory LISE and two external professors. The purpose is to create an active working environment, outside of the laboratory, for young researchers, to find relationships between different research topics, and to possible propose a research project for the laboratory.
\end{itemize}

\subsection{Experiences} 
\noindent
\tb{Internship at the laboratory of Model-Driven Engineering for Embedded Systems (LISE) - CEA}
\begin{itemize}
	\item Works are in the context of methodologies of component-based development. The mission is to model interactions
	between components based on model-driven engineering: Understanding middleware ZeroMQ and DDS, using Papyrus
	to create UML distributed application models and the tool Qompass to transform these models to code. The code
	generated runs on those middleware. 
\end{itemize}


\noindent
\tb{Project of embedded software architecture design for a SAFECRANE system}
\begin{itemize}
	\item Requirements analysis of SafeCrane system which is established to provide a safe usage of the crane. It monitors
	some environment and physical characteristics to protect humans against crane collapses. The system warns human
	agents if the crane is unbalanced wind is too strong, activates an alarm and sends a message to the base-station.
\end{itemize} 

\noindent
\tb{Software engineer for developing computer and embedded applications at Toshiba, Vietnam}: 
\begin{itemize}
	\setlength\itemsep{0.5em}
	\item Requirements analysis, software architecture and detail design, coding (language C and C++ in Visual Studio 2010),
	unit test, integration test and optimization for software part of relay protection equipment. Using IEC 60870-5-103 to
	control and communicate between smart electric devices.
	
	\item Construction of PC application to control devices by using the private protocol of Toshiba and language C\#.
\end{itemize}

\noindent
\tb{Project of design and implementation of multimedia player application} Using IDE Qt and the C++ library ffmpeg to implement a multimedia player application on Linux in Toshiba Company

