\section{A complete code generation solution for UML State Machine}
\label{sec:codegen}
This section describes a complete generation solution for UML State Machine, especially when considering the concurrency aspect. It first reminds background definitions in a formal way and some assumptions on output generated code. 
Subsection \ref{sec:thread} continues with the concurrency design for generated code based on multi-thread. 
Overview of a code generation approach based on the design is then discussed. %before a synchronization approach is introduced.

\input{sections/formalize} 

\subsection{Thread-based Concurrency for UML State Machine}
\label{sec:thread}
This Subsection describes our design for generated code. 
The design is based on an example-based analysis, which is not presented here due to space limitation.

%\subsubsection{Thread-based concurrency analysis} 

%While concurrency is an important aspect defined by the UML State machine specification, especially hierarchical and concurrent state machines with \ti{doActivity}s for states, most of existing approaches do not take into account. This is non-trivial since concurrency is dynamic in UML state machine and the number of threads used for concurrency is non-deterministic.



Let us give an analysis on the state machine example in Fig. \ref{fig:example}. Assuming that \ti{Idle} is the current active state and a \ti{verifyPIN} event is coming. 
The \ti{doActivity} behavior of \ti{Idle} \ti{doActivity(Idle)} (if has) is terminated, \ti{exit(Idle)} and the \ti{effect(t2)} (\ti{T2Effect}) are executed sequentially. 
These actions are run in a state machine main thread which reads incoming events from a "first in, first out" (FIFO) priority queue. 
Fig. \ref{fig:threading1} shows the activity diagram representing the concurrency processing \ti{verifyPIN}, in which each partition represents a thread. 
\ti{effect(t3)} and \ti{effect(t3)} are run concurrently in two threads \ti{T3Run} and \ti{T4Run}, respectively, since the transitions owning these effects outgo from a fork pseudo state. 
The entry action \ti{entry(Verifying)} is executed following the termination of the two threads. 

UML says that \ti{doActivity(Verifying)}, \ti{entry(VerifyingCard)} and \ti{entry(VerifyingPIN)} should be concurrently executed upon the completion of \ti{entry(Verifying)}, which is represented by a fork node, in which a \ti{Start} signal is sent to \ti{VerifyingDoRun} in order for commencing \ti{doActivity(Verifying)}. 
The executions of the \ti{doActivity}s of the states \ti{VerifyingCard} and \ti{VerifyingPIN} are also concurrent. 
Also, upon the completion of \ti{entry(VerifyingCard)} and \ti{entry(VerifyingPIN)}, the main thread completes the processing of the \ti{verifyPIN} event, reads next events from the queue or waits for next event occurrences.

\begin{figure}
	\centering
	\includegraphics[clip, trim=1.0cm 1.6cm 1.6cm 1cm, width=1.03\columnwidth]{figures/ThreadingExample.pdf}
	\caption{Concurrency of the ATM when receiving the \ti{verifyingPIN} event} 
	\label{fig:threading1}
\end{figure}

If no event is coming, and \ti{doActivity(VerifyingCard)} and \ti{doActivity(VerifyingPIN)} are long actions (e.g. forever loops inside), the state machine remains its active configuration and three concurrent actions including \ti{wait for events}, \ti{doActivity(VerifyingCard)}, and \ti{doActivity(VerifyingPIN)} are permanently run.

The termination time of \ti{doActivity(VerifyingCard)} and \ti{doActivity(VerifyingPIN)} is non-deterministic. 
However, a completion event is generated and pushed to the event queue whenever one of the two completes. 

\begin{comment}
For illustration, assuming that \ti{doActivity(VerifyingCard)} terminates before \ti{doActivity(VerifyingPIN)}. 
As in Fig. \ref{fig:threading2}, the Main thread checks the \ti{CompletionEvent}. \ti{exit(VerifyingCard)} and \ti{effect(t5)} are then executed sequentially. If \ti{cardValid} is computed as true as the result of \ti{doActivity(VerifyingCard)} and \ti{exit(VerifyingCard)}, the Main thread simply executes \ti{effect(t6)} and \ti{entry(CardValid)} before waiting for other events.

In contrast, Main sends a signal to stop \ti{doActivity(VerifyingPIN)} and \ti{doActivity(Verifying)}, executes exit, transition and entry actions in an appropriate order (see Fig. \ref{fig:threading2}) and waits for other events.

\begin{figure}
\centering
\includegraphics[clip, trim=1.5cm 1.6cm 1.6cm 1.2cm, width=1.03\columnwidth]{figures/ThreadingExample2.pdf}
\caption{Concurrency of the ATM when \ti{doActivity} of \ti{VerifyingCard} completes before that of \ti{VerifyingPIN}}
\label{fig:threading2}
\end{figure}
\end{comment}

We see that the number of concurrent actions is not constant but changes timely. 
Each action can either deterministically or non-deterministically terminate. 
In this sense, deterministic actions (DAs) prevent the Main thread from going to the waiting-for-event point. 
In other words, pending events in the queue are only read and processed once all deterministic actions complete. Therefore, we re-define the run-to-completion paradigm of UML state machine as following:
 
\begin{definition}
	Run-to-completion means that, in the absence of exceptions or asynchronous destruction of the context	class object or the state machine execution, a pending Event occurrence is dispatched only after the completion of all deterministic actions commenced by the processing of the current event. 
	At this point, a stable state configuration has been reached
\end{definition}

%In the example, some of DAs are as followings: \ti{effect(t2)}, \ti{effect(t4)}, \ti{entry(Verifying)}, \ti{entry(VerifyingCard)}, and non-deterministic actions (NDAs) as followings: \ti{doActivity(Verifying)}, \ti{doActivity(VerifyingCard)} and \ti{doActivity(VerifyingPIN)}.




\subsubsection{Thread-based design of generated code}
The concurrency of concurrent UML State Machines is based on multi-thread, in which there are permanent and spontaneous threads. 
While permanent threads (PTs) are created once and live as long as the state machine is alive, spontaneous threads (STs) are spawned in active for a while. 
%The method associated with a permanent thread is a non-deterministic action, which is run in parallel with the main thread. The latter reads and dispatch events from the event queue. 
Each PT is initialized at the state machine initialization. 
%The number of threads associated with NDAs is therefore equal to that of the NDAs.
The design of threads is based on the thread pool pattern, which initializes all threads at once, and the paradigm "wait-execute-wait". 
In the latter, a thread \tb{waits} for a signal to \tb{execute} its associated method and goes back to the \tb{wait} point if it receives a stop signal or its associated method completes. 
Each PT is associated with one of the following actions:
\begin{itemize}
	\item \ti{doActivity} of each state if has. %The number of \ti{doActivity} $n_{do} = \#\{s \in V|\exists doActivity(s)\}$
	
	\item Sleep function associated with a \ti{TimeEvent} which counts ticks and emits a \ti{TimeEvent} once completes.%: $n_{te} = \#\{e \in E|\ti{e is a time event}\}$.
	
	\item Change detect function associated with a \ti{ChangeEvent} which observes a variable or a boolean expression and pushes an event to the queue if changes happen.%: $n_{che} = \#\{e \in E|\ti{e is a change event}\}$.
	
	\item State machine main thread,which reads events from the event queue, and sends start and stop signals to these initial threads.
\end{itemize} 

%Therefore, The number of initial threads is $n_{threads} = n_{do} + n_{te} + n_{che}$ plus a main thread, which reads events from the event queue, and sends start and stop signals to these initial threads. 

Now we consider STs which are spawned by a parent thread, joined until and destroyed once the associated methods complete. 

\begin{comment}
The followings describe different types of DAs:

\begin{itemize}
	\item Actions executed when entering/exiting an orthogonal region, which can be: execute a chain of transition effects contained by the region before entering a stable sub-state or exiting the region completely.%: $n_{region threads} = \#\{r \in \mathcal{R}|ctner(r).kind=conc\}$
	
	\item Effects of transitions outgoing from a $fork$ and those incomings to a $join$.%: \\
	%$\mathcal{J} = \{v \in V|v.kind=join\}$ \\
	%$\mathcal{F} = \{v \in V|v.kind=fork\}$ \\
	%$$n_{FJ\_threads} = \sum_{v \in \mathcal{F}} {\#T_{outs}(v)} + \sum_{v \in \mathcal{J}} {\#T_{ins}(v)}$$.
\end{itemize}
\end{comment}

The STs follow a paradigm in which if a thread $parent$ spawns a set of threads $children$, $parent$ must wait until $children$ complete their associate methods. These threads are spawned in one of the following cases:

\begin{itemize}
	\item A thread is created for each effect of transitions' outgoing from a \ti{fork} or incoming to a \ti{join}.
	
	\item Entering a concurrent state $s$, after the execution of $entry(s)$, a thread is also created for each orthogonal region. 
	
	\item Exiting a concurrent state $s$, before the execution of $exit(s)$, a thread is also created for each region to exit the corresponding active sub-state.
	
	%\item An event is processed by active states, in which a thread per orthogonal region. 
\end{itemize}

\subsubsection{Deadlock avoidance}
Each PT is associated with a mutex for synchronization communication in the multi-thread-based generated code. 
The mutex must be locked before the method associated with the thread is executed. 
The mutex associated with the main thread preserves the run-to-completion semantics since some event such as \ti{CallEvent} can be processed synchronously and some asynchronously. Each event processing must lock the main mutex before executing the actual processing. 
%Deadlock is one of the main issues in designing multi-thread applications, in which two competing actions wait for the other to finish. In our case, 


\begin{comment}
\begin{tabular}{p{4.0cm}|p{4.0cm}}
Example code generated for doActivity  &  Option  2\\
\begin{lstlisting}[language=C++]
void doActivity(int stateId) {
  isStarts[stateId] = false;
  while(true) {
    mutex[stateId].lock();
    while(!isStarts[stateId]) {
      mutex[stateId].wait();
    }
    states[stateId].doActivity();
    isStarts[stateId] = false;
    mutex[stateId].unlock();
    if (!isStops[stateId]) {
      if (stateId == IDLE_ID || stateId == DISPENSEMONEY_ID ...) {
	    pushCompletionEvent(stateId);
	  }
    }
  }
}
	\end{lstlisting}&
	\begin{lstlisting}
	#include <stdio.h>
	
	int main()
	{
	printf("Hello world\n");
	}
	\end{lstlisting}
\end{tabular} 
\end{comment}


%\subsection{Transformation pattern}
%todo: describe the pattern

%Transformation from State machine to fUML (classes, attributes, methods)
%\lipsum[1]

\subsection{Assumption}
%todo: give some assumptions on code generation such as functions to create methods, attriutes, classes
Assuming that we want to generate from the state machine to an object oriented programming language $ActLang$, which is C++-like and supports multi-threading as following functions and resource control as mutexes.
\begin{itemize}
	%\item $genClass(n, generals, itfs)$ creates a class with its name, parent class set, and implemented interfaces as \ti{n}, \ti{generals}, and \ti{itfs}.
	
	%\item $genMtd(n, c, type, params)$ creates a method $m$ with its name as $n$ inside the class $c$, its return type as $type$, and $params$ as its parameter set.
	
	%\item $genAttr(n, c, type, multiplicity)$ creates an attribute named $n$ in the class $c$ and typed by $type$. The create attribute is an array if $multiplicity > 1$, otherwise a simple attribute.
	
	%\item $genEnum(n)$ and $genEnumLit(enum, n)$ create an enumeration and its enumeration literal, respectively.
	
	%\item $genBody(m, body)$ adds a body to a method. The body is a string which contains a list of statements.
	
	%\item $createParalle(t, seg)$ generates a mechanism which allows the segment code $seg$ run in a thread $t$. Similarly, $genWait(t), genJoin(t)$.
	
	\item A mutex has three methods $lock$, $unlock$, and $wait$, which automatically unlocks the mutex and waits until it receives a signal.  
	
	%\item $synchronize(seg)$ generates a mechanism which allows the segment code $seg$ run safely (can be either based on \ti{POSIX pthread} or \ti{Java synchronize} mechanism).
	
	%\item $toString(stts)$ is used to convert a list of statements $stts$ into a readable string which can be add to a method as its body.
	
	%\item Concatenation of two strings $str1$ and $str2$ is concisely described as $str1 + str2$.
	
	%\item \ti{WHILE}, \ti{FOR} \ti{IF}, \ti{ELSE} are symbols representing while and for loops, if and else statements.
	
	\item \ti{FORK(func)} creates a thread (lightweight process) associated with the function/method \ti{func} and \ti{JOIN(theThread)} waits until the method associated with the thread \ti{theThread} completes.
\end{itemize} 

\subsection{Code generation pattern}

\subsubsection{State transformation}
%Suppose that we want to generate a state machine $sm$ whose states are listed by $lstates$. 
A common state interface $IState$ is created. The interface contains three methods, namely, \ti{entry}, \ti{exit}, and \ti{doActivity} respectively corresponding to three state actions. To preserve the hierarchy of composite states, the interface also has two attributes called \ti{actives} and \ti{previousActives} referring to current and previous active sub-states in case of the presence of history states, and a list of deferred event identifiers.

Each UML state is transformed into an instance of the interface associated with a state ID (which is a child element of an enumeration) inside the active class $C$. 
During initialization, each instance delegates its methods to suitable implementation, e.g. function pointers in C++. 
%In other object-oriented languages such as Java, this is done with anonymous sub-classes of the interface. 
For example, Listing \ref{lst:IStateCpp} in Appendix shows the interface and its instances. 
%The actions of the states are named depending on the name of the states. In the following sections, we only consider \ti{ActLang} as a C++-like. The discussion of other object-oriented languages are much similar since these share the same concepts.

Each \ti{doActivity} is associated with a permanent thread and a mutex. The \ti{doActivity} thread is initialized, waits for a start signal, executes the \ti{doActivity} code, generates a completion if the state is atomic and still active, and goes back to the waiting point as the paradigm above. Listing \ref{lst:doActivity} in Appendix shows a code segment for \ti{doActivity} threads. 




\begin{comment}
\begin{lstlisting}[mathescape=true, caption=IState interface and annonymous sub-classes in Java, label=lst:IStateJava, frame=single, language=JAVA]
public interface IState {
  public IState[] pres; 
  public IState[] actives;
  public EventId defEvents;
  public void entry();
  public void exit();
  public void doActivity();
}
class C {
private IState states[NUM_STATES];
public C() {
  states[S0_ID] = new IState() {
    public void entry() {
      S0_entry();
    }
    ...
  }
}
public void S0_entry() {...}
}
\end{lstlisting}
\end{comment}

\begin{comment}
The procedure to generate the code for states is shown in Listing \ref{lst:procedure1}. It first creates the state interface $IState$ (in C++, it is either a class or a struct). The array attribute is then created with the number of states as its size. Each state is also associated with a state ID which is a child of an enumeration. Finally, the constructor of $C$ is created to initialize and make methods of the attribute instances refer to \ti{entry/exit/doActivity} action methods of $C$. The implementation of action methods in the context class $C$ is similar to the delegation pattern proposed by the authors in \cite{Niaz2004} but dramatically decreases the memory consumption since only one common interface for all states is created instead of a class for each state in \cite{Niaz2004}.

\begin{lstlisting}[mathescape=true, caption=Procedure to create code for states, label=lst:procedure1, frame=single]
IState = genClass('IState', $\emptyset$, $\emptyset$);
stateIdEnum = genEnum('StateIdEnum');
foreach s in lstates
  genEnumLit(stateIdEnum, s.name + '_ID');
  mtd = genMtd(s.name + '_entry', C, 
			null, null);
  genBody(mtd, toString(entry(s)));
  ...
genEnumLit(stateIdEnum, 'NUM_STATES');  
genAttr('states', C, IState, NUM_STATES); 
genMtd(C.name, C, null, null);
\end{lstlisting}
\end{comment}



\subsubsection{Region transformation}
\label{subsubsec:region-trans}
Each region is transformed into an entering and exiting method. 
While the entering method controls how a region $r$ is entered from an outside transition $t$ ($src(t) \notin vertices(r)$), the exiting method exits completely a region by executing exit actions of sub-states from innermost to outermost.

\begin{figure}
	\centering
	\includegraphics[clip, trim=0.2cm 0.2cm 0.2cm 0.2cm, width=1.0\columnwidth]{figures/EnteringStateExample.pdf}
	\caption{Example illustrating different ways entering a composite state} 
	\label{fig:entering}
\end{figure}

A region $r$ is entered by either a transition $t$ ending at the border of its containing state or on a sub-vertex (direct or indirect), depending on how the state machine is designed. 
The following lists different ways $r$ may be entered:
\begin{itemize}
	\item Way 1: entering by default: $tgt(t) = owner(r) \wedge src(t) \notin vertices(r)$.
	
	\item Way 2: entering on a direct sub-vertex: $tgt(t) \in vertices(r) \wedge src(t) \notin vertices(r)$.
	
	\item Way 3: entering on an indirect sub-vertex: $owner(tgt(t)) \in vertices^+(r) \wedge src(t) \notin vertices(r)$.
\end{itemize} 

All of the entering ways execute the entry action of the containing composite state $entry(owner(r))$ after $effect(t)$. \ti{doActivity(owner(r))} is then signaled to be run in its associated thread. The actions afterwards are different for each way. To illustrate, we use an example as in Fig. \ref{fig:entering} with \ti{S1} as a target composite state. \ti{t1}, \ti{t3} and \ti{t6} are in the ways of 1 and 3, respectively, while \ti{t2, t5} in the way 2. 

The entering method associated with the region $r$ of \ti{S1} has a parameter $enter\_mode$ indicating how actions should be executed. The number of modes depends on the number of transitions coming to the composite state $S1$ specified as: $\#modes(s) = \\ \#\{v \in vertices(s)| v.kind=initial\} + \\ \#\{v \in vertices(s)| \exists t \in T_{ins}(v), src(t) \notin vertices(s)\} + \\ 
\#\{v \in vertices^+(s)\setminus vertices(s)|\exists t \in T_{ins}(v), src(t) \notin vertices^+(s)\}$. The detail of how these modes are implemented in specific languages are not discussed here. The readers are recommended to read the example in Listing \ref{lst:region} in Appendix. 

 

%For each value in $values(s)$, the region of $S1$ is entered and executes different actions. 

%Entering on a direct sub-state (\ti{S2}) sets the active sub-state of \ti{S1} directly to \ti{S2}. 
%In case of an indirect sub-state ($S4$), the entry action of $S3$ is executed before $S4$ is set as the active-sub state of $S3$ and the execution of $entry(S4)$. 


Transitioning from a vertex to a pseudo-state of the composite state (transition from $S0$ to $SH$ is a particular case) is not as simple as that of two states. It needs a systematic approach which generates code for a transition outgoing from a vertex to any other one. This is detailed in the next section.

\subsubsection{Event and transition transformation}
\label{subsec:event}
\paragraph{Events}
An event enumeration \ti{EventId} is created whose children are event identifiers associated with events. 
Each event $e$) is also transformed into a method $mtd_e$ in the context class $C$. 
%Suppose $levents$ is the list of events which can be processed by the state machine $sm$. 
Besides the explicitly defined events of the state machines, the event list contains a special event called $CompletionEvent$, which is implicitly implemented as an asynchronous \ti{CallEvent}. %The latter is, following the UML specification, an implicit event triggering triggerless transitions. It is emitted when either \ti{doActivity} of an atomic state finishes its execution or all regions of a composite state have reached to a final state. 
For each event type, the transformation is realized as followings:

\begin{itemize}
	\item \ti{CallEvent} $ce$: The associated operation $op(ce)$ can be either synchronous or asynchronous. When $op(ce)$ is called, it waits and locks the main mutex protecting the run-to-completion semantics, and executes $mtd_{ce}$. Contrarily, the parameters of the asynchronous operation are used to create a signal which is transformed similarly to the case of $SignalEvent$.
	
	\item \ti{SignalEvent} $se$: $SignalEvent$ is asynchronous. 
	The signal associated with $se$ is written into the event queue of the active class $C$ by an operation which takes as input the signal. 
	
	\item \ti{TimeEvent} $te$: A thread $teThread$ associated with $te$ is created and initialized at the initialization of the state machine. 
	Within the execution of $teThread$, the method associated with $te$ waits for a signal, which is sent after the execution of the entry of a state $s \in \{v \in V|\exists t \in T_{outs}(v), te \in events(t)\}$, to start sleeping for a duration $d$ of $te$. 
	When the duration expires, $te$ is emitted and written to the event queue if $s$ is still active.
	
	\item \ti{ChangeEvent} $che$: Similar to $TimeEvent$, a thread $cheThread$ is initialized at initialization but the associated method $mtd$ does not wait for a signal to start. $mtd$ periodically checks whether the value of the associated boolean expression $ex(che)$ changes by comparing the current value with the previous value. 
	If a change happen, $che$ is committed to the event queue.
	
	\item \ti{Any}: any of the above events can trigger the associated transitions.
\end{itemize}

As above presented, all asynchronous incoming events are stored in a FIFO priority queue, in which each event type has a configurable priority. $CompletionEvent$ always has the highest priority. 
Others are equal by default.

\paragraph{Transitions}

%To process events, for each event, a method is implemented in $C$. 
Each event is associated with a list of transitions. We suppose $T_{trig}(e)$ is the transition list, which can be triggered by the event $e$, and $S_{trig}(e) = \{src(t) | t \in T_{trig}(e)\}$. In other words, $S_{trig(e)}$ is a set of states which are the source states of the transitions in $T_{trig}(e)$. To present how the body of event methods is generated, we define functions as followings:
\begin{itemize}
	\item Vertex depth $dp(v)$ is defined as: $dp(v) = 1$ if v is a root vertex, otherwise $dp(v) = dp(owner(v)) + 1$.
	
	\item $Map_{e}(s) \subset S_{trig(e)} | \forall sub \in Map_e(s): owner(sub) = s$, $Prt(e) = \{s \in V| Map_{e}(v) \neq \emptyset\}$. $Prt(e)$ is an ordered list whose length is $len(Prt\{e\})$ and elements are accessed by index ($get$). The order of $Prt(e)$ is defined as:	$\forall i, j \leq len(Prt\{e\})$, 
	\\ if $i < j, dp(Prt(e).get(i)) \geq dp(Prt(e).get(j))$. 	
\end{itemize}

The procedure in Listing \ref{lst:eventproc} describes how the generation process works with an event. 
It first finds the innermost active states which are able to react $e$ by orderly looping over $Lm_e$. 
For each transition outgoing from an innermost state, code for active states and deferral events, guard checking and transition code segments are generated by $GENERATE\_STATE\_EVENT\_CHECK$, $GENERATE\_GUARD(t)$ and \ti{GENTRANS}, respectively. 
If the identifier of $e$ is equal to one of the events listed in $defEvents$ of the corresponding state (not shown in this paper), it is deferred by putting it to a deferral event queue managed by the main thread, which also pushes the deferred events back to the main queue once one of the pending events is processed. 

\begin{lstlisting}[mathescape=true, caption=Generation process for an event, label=lst:eventproc, frame=single][H]
$\forall$ item $\in Lm(e)$
  $\forall s \in Map_e(item)$
    $T_s = \{t \in T_{trig}(e)|src(t) = s\}$
    $\forall t \in T_s$
      $GENERATE\_STATE\_EVENT\_CHECK(s, t, e)$
      $GENERATE\_GUARD(t)$
      $GENTRANS(s,t,tgt(t))$  	
\end{lstlisting}

Depending on the target of $t$, $GENERATE\_STATE\_EVENT\_CHECK$ can generate single or multiple active state checking code. 
The latter occurs if $tgt(t)$ is a $join$. 
The detailed discussion on these is not presented due to space limitation. Listing \ref{lst:event1}, line 1-2, shows an example for multiple checking.   

Generally, \ti{GENTRANS} generates code for transitions between any vertexes satisfying the constraints described in Section \ref{subsec:background}. The detail of \ti{GENTRANS} is not presented here. Each pseudo state is transformed as the followings:
%Algorithm \ref{alg:transitiongeneration} shows how the transition code generation works. The generated code is bounded by the deferral events, active states, and guard checking.

\begin{comment}
\begin{algorithm}[]
	\caption{Code generation for transition
		\label{alg:transitiongeneration}}
	\begin{algorithmic}[1]
		\Require{A source $v_{s}$, a target vertex $v_{t}$ and a transition $t$}
		\Ensure{Code generation for transition}
		\Procedure{genTrans}{$v_s$, $v_t$, $t$}
		\Let{$H_s$}{$v_s \cup owner^+(v_s)$}
		\Let{$H_t$}{$v_t \cup owner^+(v_t)$}
		\State {$s_{ex} \in H_s, s_{en} \in H_t | owner(s_{ex}) = owner(s_{en})$}
		%\Let{\{$s_{ex}, s_{en}$\}}{$FINDEXE(v_s, v_t)$}
		\State {//Generate IF-ELSE statements for junctions}
		\If {$s_{ex}$ is a state}
		\For {$ r \in regions(s_{ex})$}
		\State {$FORK(RegionExit(r))$}
		\EndFor
		\State {//Generate JOIN for threads created above}
		\State {//Generate sendStopSignal to $s_{ex}$}
		\State {$exit(s_{ex})$}	
		\EndIf
		\If {$v_t.kind=join$}
		\For {$in \in T_ins(v_t)$}
		\State {$FORK(effect(in))$}
		\EndFor
		\State {//Generate JOIN for threads created above}
		\Else
		\State {$effect(t)$}	
		\EndIf
		\If {$s_{en}$ is a state}
		\State {$entry(s_{en})$}
		\State {//Generate sendStartSignal to $s_{en}$}
		\If {$s_{en}.kind\in\{conp,conc\}$}
		\For {$ r \in regions(s_{en})$}
		\State {$FORK(RegionEnter(r))$}
		\EndFor
		\State {//Generate JOIN for threads created above}
		\EndIf
		\Else
		\State {//Generate for pseudo states by patterns}
		\EndIf
		\EndProcedure	
	\end{algorithmic}
\end{algorithm}
\end{comment}

%In the first place, Algorithm \ref{alg:transitiongeneration} looks for the composite states $s_{ex}$ and $s_{en}$ at the highest level to be exited and entered (line 2-4), respectively. 
%If the transition $t$ is part of a compound transition, which involves some $junction$s, IF-ELSE statements are generated first (as PSCS says $junction$ is evaluated before any action). 
%The composite state is exited by calling the associated exiting region methods (FORK and JOIN for orthogonal regions) and followed by the generated code of transition effects. Entering region methods are then called once the above code completes the execution. If the target $v_t$ of the transition $t$ is a pseudo state, the generation algorithm corresponding to the pseudo state type is called. These algorithms are shown as the below list. 
 

%It generates the code checking for active states respecting the UML semantics in which the innermost states process the incoming event first. To do this, it first looks in the source state list $S_{trig(e)}$ for the innermost states that accept the event triggering its outgoing transitions. If these found states are children of a concurrent state, $genStateCheck$ generates the checking codes run in parallel, which will be described later in \ref{subsubsec:thread}. Otherwise said, sequential code is generated.


\begin{comment}



\begin{algorithm}[]
	\caption{Find states should be exited and entered
		\label{alg:findexit-entry}}
	\begin{algorithmic}[1]
		\Require{A source $v_{s}$ and a target vertex $v_{t}$}
		\Ensure{Vertexes $s_{ex}$, $s_{en}$ to be exited, and entered, respectively}
		\Procedure{findExE}{$v_s$, $v_t$}
			\Let{$H_s$}{$v_s \cup owner^+(v_s)$}
			\Let{$H_t$}{$v_t \cup owner^+(v_t)$}
			\State {$s_{ex} \in H_s, s_{en} \in H_t | owner(s_{ex}) = owner(s_{en})$}
		\EndProcedure	
	\end{algorithmic}
\end{algorithm}
\end{comment}

\begin{itemize}
	\item $join$: Use $GENTRANS$ for $v$'s outgoing transition.
	
	\item $fork$: Use $FORK$ and $JOIN$ for each of outgoing transitions of $v$.
	
	\item $choice$: For each outgoing, an $IF-ELSE$ is generated for the guard of the outgoing together with code generated by $GENTRANS$ (see Listing \ref{lst:event1}).
	
	\item $junction$: As a static version $choice$, a $junction$ is doubly evaluated. The first evaluation is before any action executed in compound transitions (see Listing \ref{lst:event1}). The output value of the first evaluation is used to determine which transition outgoing from $junction$ is taken in a second evaluation. %is transformed into an attribute $junc_attr$ and evaluated before . 
	%The value of $junc_attr$ is then used to choose the appropriate transition at the place of $junction$.
	
	\item \ti{shallow history}: The identifiers of states to be exited are kept in $pres$ of $IState$. Restoring the active states using the history is exampled as in Listing \ref{lst:region}. The entering method is executed as default mode at the first time the corresponding composite state is entered (see Listing \ref{lst:region}). The previous active sub-states are updated by saving the active state identifier to \ti{previousActives} before exiting the region containing the history.
	
	\item \ti{deep history}: Saving and restoring active states are done at all state hierarchy levels from the composite state containing the deep history down to atomic states. Updating \ti{pres} is committed before exiting the region, which is directly or indirectly contained by a parent state, in which a deep history is present.  
	
	\item $enpoint$: If $enpoint$ has no outgoing transition, the corresponding composite state is entered by default. Otherwise said, $GENTRANS$ is called to generate code for the outgoing transition.
	
	\item $expoint$: The code for the unique transition outgoing from $expoint$ is generated by using $GENTRANS$.
	
	\item $terminate$: The code executes the exit action of the innermost active state, the effect of the transition and destroys the state machine object.
\end{itemize}

%\subsubsection{Example Code} Listing \ref{lst:event1} shows a code segment generated. %for the processing of $verifyingPIN$. Single checking (line 1) checks whether $Idle$ is the current active state, in which $activeStateID$ is the identifier of the current root active state. The $doActivity$ behavior of $Idle$ is then stopped upon receiving a stop signal. The effect of $t2$, $effect(t3)$ and $effect(t4)$ are then executed after $exit(Idle)$. The execution of $emtry(Verifying)$ then follows the changing of root active state to $Verifying$. $doActivity(Verifying)$ is triggered and followed by concurrently entering the two orthogonal regions of $Verifying$ with appropriate modes.

 \begin{comment}
\begin{lstlisting}[caption=Example code generated for event $verifyingPIN$, label=lst:event, language=C++]
if (activeStateID == IDLE_ID) {
  sendStopSignal(IDLE_ID);  exit_Idle();
  effect_t2();
  thread_t3 = FORK(effect_t3);  thread_t4 = FORK(effect_t4);
  JOIN(thread_t3);  JOIN(thread_t4);
  activeStateID = VERIFYING_ID;  entry_Verifying();
  sendStartSignal(VERIFYING_ID);
  th_r1 = FORK(R1Enter(VERIFYINGCARD_MODE));
  th_r2 = FORK(R2Enter(VERIFYINGPIN_MODE));
  JOIN(th_r1);  JOIN(th_r2);
}
\end{lstlisting}
\end{comment}

%The discussion of Listing \ref{lst:event1} is similar to Listing \ref{lst:event} except that a multiple checking is executed (line 1-2) instead of a single one. The evaluation for $Junction1$ is executed (line 4) before any other actions (semantic conformance) to decide the decision should be taken (line 10-17). 





%\tb{Limitations}


 


 

