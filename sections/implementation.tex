\section{An Implementation: synchronizing UML models and C++ code}
\label{sec:implementation}

The proposed model-code synchronization approach can be automated.
We examine an Eclipse-based implementation of the IDE proposed in
Section \ref{sec:use-cases_ide}. The implementation is used in the synchronization processes
proposed in Section \ref{sec:processes}.
Our implementation targets synchronization of Object Management Group (OMG) standard UML 2 and C++11 code.

The use-cases of the IDE are implemented with some Eclipse technologies. More information
on the technologies is available on the Eclipse Projects website \cite{eclipse_foundation_eclipse_2016}.

\textbf{Eclipse CDT} is an IDE for C++ development. It is used in the \texttt{Edit Code} use-case.
Eclipse \textbf{Papyrus} is used for the \texttt{Edit Model} use-case.
Papyrus is an open-source UML modeler that uses the Eclipse Modeling Framework (EMF)
implementation of the OMG-standard UML 2.
Papyrus supports UML profiles for domain-specific modeling, and we use the UML profile of Papyrus
dedicated to C++ to facilitate and accelerate modeling of some C++ features.
%Furthermore, in standard UML there are elements like OpaqueBehavior which
%allow to express programming language specifics in the model.

We developed plugins for Papyrus to \texttt{Generate Code} from UML to C++ and to \texttt{Reverse Code}.
The batch modes of these use-cases do not need additional technologies to implement.
For use-cases \texttt{Generate Code (Incremental)} and \texttt{Reverse Code (Incremental)}, we choose to
listen to modification events in the model and code respectively. Listening to modification events
is one possible approach in incremental model transformation \cite{kusel_survey_2013}.
The \textbf{Viatra}
API is used to listen to such events in the model. The Eclipse CDT API is used
to listen to modification events in the code. These list of events are used to
either generate code or reverse code incrementally.

\textbf{EMF Compare} is used for the \texttt{Synchronize Model} use-case when
models are implemented with EMF.
We adapted EMF Compare in order to synchronize UML models for
our specific work.
Eclipse CDT is used for the \texttt{Synchronize Code} use-case with its built-in C++ features.



%Figure \ref{fig:strategies_impl} shows how technologies are combined
%for UML-C++ synchronization, within a same Eclipse IDE. The synchronization strategies,
%in the scenario where model and code are concurrently edition, is taken as an example to
%show how the technologies are involved in the synchronization processes.
%
%\begin{figure}
%\centering
%\includegraphics[width=\columnwidth]{figures/strategies_impl}
%\caption{Implementation of synchronization strategies}
%\label{fig:strategies_impl}
%\end{figure}

The next section describes some experiments performed using the above implementation
of our synchronization solution.