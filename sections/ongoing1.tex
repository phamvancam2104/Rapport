\section{Ongoing and future work}
\label{sec:ongoing-future}
\subsection{Towards synchronization of UML State Machine and programming languages}
%\subsubsection{Problem statement}
%\begin{itemize}
In Section \ref{sec:approach}, an approach for round-tripping USMs and code is presented. 
However, the approach only support a sub-set of USM concepts. 
Furthermore, the size of binary file compiled from the generated code is larger than existing approaches using SWITCH/IF constructions (15\% but not detailed here). 
This is not suitable to applications to be deployed on resource-constrained systems.
	
Synchronization of UML State Machine and code is hard, even impossible in case of resource constraints since one-to-one mappings from UML State Machine concepts to code statements are hard to achieve. 
To illustrate, Fig. \ref{fig:sm-sync} \tb{(m1)} and \tb{(c1)} show a USM example and its code generated by our code generator (see Section  \ref{subsec:usm2code}). 
The example contains some pseudo states such as exit point, choice, and history. 
These are supported by only a few tools such as IBM Rhapsody and ours. 
It is worth noting that it is not trivial to understand the USM control flow in the generated code. 
Hence, programmers feel hard, even impossible, to modify the flow in the code.

For example, for some reasons, the USM should be evolved to the next version shown in Fig. \ref{fig:sm-sync} \tb{(m2)}.
The new version removes the exit point, changes the shallow to deep history pseudo state, and adding some states. 
The code should therefore be modified to \tb{(c2)}.
As discussed in Section \ref{sec:introduction}, this modification should be able to be realized by either programmers or using a modeling tool with a built-in code generator because of many mentioned reasons for collaboration. 
However, on the one hand, modifying \tb{(c1)} to \tb{(c2)} is non-trivial and includes much effort for testing. 
On the other hand, the modeling tool is usually expensive and not suitable for small enterprises (that's why many surveys on the use of MDE in practice only focus on large companies even though many advantages of MDE such as productivity and maintainability are suitable to small companies).


%\end{itemize}


\begin{figure*}
	\centering
	\includegraphics[clip, trim=1.8cm 3.5cm 0.5cm 3.2cm, width=\linewidth]{figures/sm-sync}
	\caption{State machine and code evolution example} 
	\label{fig:sm-sync}
\end{figure*}


%\subsubsection{Idea overview}
%\paragraph{\tb{General idea}} ~\\
Our goal is to provide a mechanism from which software engineering stakeholders can profit in practice, e.g. software architects work with diagram-based languages while programmers are productive with text-based languages. 
%This study is the realization of the approach in Fig. \ref{fig:smapproaches}.
To do it, we introduce newly additional constructs to existing languages. 
The former are conformant to the latter. 
Specifically, we use C++ preprocessors to seamlessly connect USM features to the code level. 
The idea is inspired by ArchJava presented in \cite{Aldrich:2002:ACS:581339.581365}, in which the authors proposed an approach for co-evolution of architecture and implementation by providing concepts such as components and ports to Java.

\begin{figure}
	\centering
	\includegraphics[clip, trim=0cm 12cm 4.5cm 0cm, width=\linewidth]{figures/frontend}
	\caption{Approach for synchronization of USMs and code} 
	\label{fig:fontend}
\end{figure}



%\begin{minipage}{0.95\columnwidth}
\begin{lstlisting}[language=C++, caption=Final code after synchronization, label=lst:frontendEx]
#include "SMDefinitions.h"
class System {
	STATE_MACHINE(Machine) {
		INITIAL_STATE(S1, NULL) {
			INITIAL(Initial1);
			STATE(S11) {
				STATE(S111) {};
			};
			CHOICE(c1);
			EXIT_POINT(ex1);
		};
		STATE(S2) {
			SHALLOW_HISTORY(h1);
			FINAL_STATE(S21);
		};
		FINAL_STATE(S3);
		//Event table definitions
		//Transition table
		TRANSITION_TABLE {
			TRANSITION(S111, c1, NULL, NULL, NULL);
		}
	};
	//class member declarations
};
\end{lstlisting}
%\end{minipage}


In our approach, we use C++ macros to define USMs in a front-end generated code. 
Instead of directly generating the runnable code, we derive a C++ front-end code (see Fig. \ref{fig:fontend}), which is easy for C++ developers to integrate and modify USM in code effectively. 
Listing \ref{lst:frontendEx} shows the C++ version with additional constructs of the USM example in Fig. \ref{fig:sm-sync} \tb{(m1)}.
The constructs are written in the same source file of existing languages and  synchronized with UML State Machine in diagrammatic forms by using the generic pattern proposed in Section \ref{sec:processes}. 
Generated code with full UML State Machine features is achieved by using a comprehensive generation solution for USMs, which is not detailed in this report due to space limitation.
%by the model-oriented programming language \ti{Umple}\footnote{Umple, \url{http://cruise.eecs.uottawa.ca/umple/}}. 
%The purpose of the proposed language is to provide a text-based DSL compliant to the UML State Machine specification, which is not correctly defined by Umple.  

The additional constructs allow programmers productively write \tb{UML-compliant and complete state machines} without restrictions in a textual way. 
\tb{UML-compliance and completeness} focus on the full support for UML State Machine features and the semantics of generated code. 
This approach \ti{adapts USMs to existing programming languages}, which differs from other text-based modeling such as Umple\footnote{Umple, \url{http://cruise.eecs.uottawa.ca/umple/}} and ThingML\footnote{ThingML, \url{http://thingml.org/}}. The latter follow: \ti{adapting existing languages into USMs}. 
%and are key characteristics used to differentiate this proposed language from other non UML-compliant languages such as Umple and . 
 
%This language is a convergence of the generic pattern and the complete code generation to provide an efficient and productive way to develop reactive, real-time and embedded systems.


\begin{comment}
\paragraph{\tb{Benefits}} ~\\
Followings are benefits of this proposition:
\begin{itemize}
	\item Software architects are freely to describe the behavior of systems by using state machine diagrams.
	
	\item Programmers are freely to work with a text-based environment to be productive.
	
	\item State machines used for generating code are not restricted to simple or non-concurrent cases as in other approaches.
	
	\item At anytime, generated code and software architects-built diagrams are consistent by using the generic synchronization pattern.
\end{itemize}
\end{comment}


\begin{comment}
\subsection{Model-generated code co-debugging: A Papyrus Debugger}
Debugging is one of the most important activities in software development to find and fix bugs. 
In MDE, when having bugs in generated code, it is difficult to locate the code statements or the model elements, which have the bugs. 
MCCD enables modelers/developers to set breakpoints on model elements/code statements, respectively, and the breakpoints are also set on the associated code elements/model elements.
Thus, MCCD helps seamlessly collaborate model and code in debugging.
Current tools and approaches \cite{moka} support the model execution and debugging, especially for UML Activity and State Machine Diagrams, using Java/fUML as underlying action language.
%However, model-code co-debugging 
\end{comment}


\subsection{Synchronization of generated and user-modified code with template evolution}
\label{subsec:sync-template}
In MDE, implementation code is generated from models using generation templates. 
When models or templates change, code needs to be changed accordingly.
However, when these artifacts are concurrently modified, for some reasons, making these consistent again are challenging. 
Some approaches using specialized comments such as \ti{@generated} and \ti{@non generated} to specify which code areas should be untouched or modified, respectively.
When model elements and code change, code areas (user-modified code) marked by \ti{@non generated} are preserved. 
It means that changes made to models or templates are not propagated to some code areas. 
This may lead to the dead code problem, in which the user-modified code is not used in the new generated code. 
Fig. \ref{fig:example} shows an example consisting of an interface, a generation template, and the corresponding generated code, which is tagged using the specialized comment \ti{@non generated code}.

Nothing remains unchanged, hence the interface, the template, and the generated code can be, for some reasons, modified as, for example, in Fig. \ref{fig:example-modified}. 
Existing approaches preserve the user-modified code marked by \ti{@non generated code}.
Therefore, the modified code in Fig. \ref{fig:example-modified} is not overwritten.
Thus, the changes made to the \ti{sum} operation of the interface and the template are not propagated to the implementation code.
Alternatively, the users can force the generator to overwrite the existing code to propagate the model and template changes to the code.
Consequently, the modifications made to code are lost due to the overwriting.
 


\begin{figure*}
	\centering
	\includegraphics[clip, trim=0cm 1cm 1.7cm 7.2cm, width=\linewidth]{figures/example}
	\caption{(a): Interface, template and generated code; (b) artifacts evolution} 
	\label{fig:example}
\end{figure*}

\begin{comment}
\begin{figure}
	\centering
	\includegraphics[clip, trim=2.0cm 7.9cm 9.9cm 1.5cm, width=1\columnwidth]{figures/example-modified}
	\caption{Interface, template and generated code are modified} 
	\label{fig:example-modified}
\end{figure}
\end{comment}



\begin{minipage}{0.95\columnwidth}
\begin{lstlisting}[language=C++, caption=Final code after synchronization, label=lst:example-final]
//@non generated
int sum(int a, int b, int c) {
	int size;
	{
		int varName = a;
		AnsiMarshall(&pBuffer, &varName);
		size += sizeof(a);
	}
	{
		int varName = b;
		AnsiMarshall(&pBuffer, &varName);
		size += sizeof(b);
	}
	{
		int varName = c;
		AnsiMarshall(&pBuffer, &varName);
		size += sizeof(c);
	}
	size = size + 3;
	send(pBuffer, size);
}

\end{lstlisting}
\end{minipage}

The goal of our approach is to propagate model and template changes while keeping the user modifications made to the code.
Our approach generates a code \ti{C1} generated from the modified model and template. 
\ti{C1} is then compared to the code \ti{C} generated from the origin model and template.
A difference \ti{diff1} is computed by the comparison.
Similarly, another difference \ti{diff2} is taken by comparing the previous code version and the modified code. 
\ti{diff1} and \ti{diff2} are then used for detecting conflicts and merging these differences to the code.
The final version of the code after synchronization is shown in Listing \ref{lst:example-final}. 

\subsection{Verification of semantic-conformance of generated code runtime execution}
UML State Machine and its visualizations are widely used for modeling and designing real-time embedded systems. Although many existing approaches can generate code from state machines, the semantic conformance of runtime execution of generated code is not verified yet. The goal of this study is to provide a verification of code generated by the above generation approach. 

Two approaches can be candidates for the verification. The first is to use bi-simulation (see Fig. 
\ref{fig:bisimu}), in which a state machine is simulated by the MOKA engine \cite{moka}, which is a model simulator and offers PSSM. The same state machine is also used for generating code, which is compiled and executed. The execution traces obtained from the simulation and execution are then compared to each other. Generated code for a given state machine is considered semantic-conformant, within the scope of PSSM, if both of the traces are identical. Existing approaches had little chance to do the verification since PSSM is very new.
The second one is to utilize model checking techniques, which model checks the generated code conforming to a specification, which is transformed from the state machine.

\begin{figure}
	\centering
	\includegraphics[trim={0.0cm 8.5cm 16.6cm 6.5cm},clip, width=\columnwidth]{figures/semanticconformance}
	\caption{Bi-simulation to test the semantic-conformance of generated code runtime execution}
	\label{fig:bisimu}
\end{figure}