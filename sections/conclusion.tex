\section{Conclusion}
\label{sec:conclusion}

Model-Driven Engineering is considered an efficient way to dealing with the complexity of systems today. 
MDE relies on two mechanisms abstraction and automation. 
The latter are realized by artifact synchronizations.  
Despite many advantages of MDE, the adoption level of MDE into industries is still not high as expected. 
In order to raise the adoption level, the core technologies of MDE need to be enhanced.   
This report presents the scope of the thesis, whose focus is to improve the core technologies of MDE such as artifact transformation, synchronization and code generation. 

The report starts with describing major topics covered and developed in the deployment of the thesis. 
Different approaches around the technologies are proposed for wider integrating MDE into industrial practice to gain software quality and productivity. The proposed approaches are designed to support a continuous collaboration between software architects and programmers allowing each to use the working practices of choice
%It can be used as a pattern
%to solve the issue of model-code synchronization when artifacts co-evolve.
%The proposition targets collaboration between software architects and programmers,
%without imposing to either developers a diagram-based language or a textual language.

Two approaches including a generic pattern for artifact synchronization and a complete generation solution for UML State Machine are chosen to be detailed in this report because of their importance. 
Other works are also mentioned in the report. 
The overview of some ongoing and future works is also described to provide readers a plan for the next steps of the thesis deployment.  

\begin{comment}
Several simulations were used to validate our synchronization approach with respect to both laws of round-trip engineering. 
In addition, a simulation of scenarios in which both the model and the code were edited concurrently was performed, further demonstrating the viability of the approach, even in such a highly dynamic scenario.

The approach was then applied to a real-world application: the Papyrus-RT runtime system. This system was
originally developed in C++ with some non-object-oriented code.
The experiment showed that the main difficulty of using our approach applied to such a system was
the need to edit and refactor the program code in such way that it can be reverse engineered into a corresponding UML model without loss of information.
Once the processes of our synchronization solution were bootstrapped in this way, dynamic collaboration
between software architects and programmers was possible. We were then able to
reap the important benefits of MDE; development was facilitated through increased automation and system maintainability and evolvability were improved.

%Realization-wise, in the future, we would like to make our implementation even more generic.
%Indeed, we had to modify tools like EMF Compare to suit the
%implementation of our solution with other Eclipse technologies.
Currently we are implementing our generic synchronization pattern for several other
programming languages, like Java. Part of our future work is to federate
implementations of the system, in heterogeneous programming languages,
as one or several models based on a common core in UML.
\end{comment}
%The work presented in this report is supported by the European project SafeAdapt, grant agreement No. 608945, see \ti{http://www.SafeAdapt.eu}. The project deals with adaptive system with additional safety and real time constraints. The adaptation and safety aspects are stored in different artifacts in order to achieve a separation of concerns. These artifacts need to be synchronized. In this project in part motivated by the perceived gap between diagram-based languages and
%textual languages, which impeded greater adoption of MDE among practitioners.

%Therefore, as a future work item, we would like to assess to what extent our
%solution helps in eliminating the unnecessary and dogmatic separation of models from code. 
%We feel that this would be useful to both the MDE community and the more traditionally-oriented software communities.
%As MDE is gradually integrated into industrial practices, correct-by-construction approaches, through MDE,
%can be adopted to deliver higher quality software.
