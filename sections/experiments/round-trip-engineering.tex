\subsection{Simulations to assess synchronization processes}
\label{sec:simulation}

Simulations have been conducted to test that the implementation of
our proposition respects the round-trip engineering
laws \cite{Foster:2007:CBT:1232420.1232424}
\ti{right-invertibility} and \ti{left-invertibility}.
These laws are stated as follows:

%The RTE takes as input the previous consistent model and code and their edited versions and outputs the updated artifacts: \ti{rte(m, c, m*, c*) = ($m_{updated}, c_{updated}$)}. A round-trip must satisfy two laws, namely, \ti{right-invertibility} and \ti{left-invertibility} stated as followings:

\noindent
\tb{Law 1}: Right-invertibility means that
not editing the code (or model, respectively) shall be reflected
as not editing the model (respectively code).
If no editions are made to the code, the model used for generating the code and the model received by
immediately reversing the generated code must be the same.

\noindent
\tb{Law 2}: Left-invertibility means that all editions
on the code are captured and correctly
propagated to the model
so that the edited code can be fully recreated by applying code generation to the updated model.

In the following sections, the model generator used for the simulations is
presented first. Then, the simulations performed for both above laws are described.
Finally, we discuss the simulation of the process defined for scenario 3 (Section \ref{sec:scenario3}), in which both the model and the code are edited concurrently.

\subsubsection{Randomly generated UML models}

In the simulations, random UML models are generated
with a configurable model generator.
The generated models contain C++ features represented via UML.
The generator can be configured to generate a desired average number of each type of C++ feature to be represented as a UML element.

Three packages are generated for each model.
Each package contains 60 classes, and, on the average
10 enumerations, 10 structures, and
10 function pointers. Class members include
methods with parameters, and attributes.

Types of parameters and attributes are chosen randomly.
Methods have randomly generated bodies that use other classes.
Attributes have randomly generated default values of the appropriate types.

Relationships between
classes are also generated: associations, inheritances, and dependencies.
When dependencies are generated, the generator enforces that the source
class of the dependency has a method that uses the target class of the dependency.

%Some parameters of the model generator are shown in Table \ref{table:info-setup}.
%
%% TODO I don't know if the table itself is useful.
%\begin{table}[]
%\centering
%\caption{Appearance probabilities and information for setting up the model generator}
%\label{table:info-setup}
%\begin{tabular}{|l|l|}
%\hline
%\rowcolor{Gray}
%Generator configuration                                   & Value            \\ \hline
%Number of packages                                        & 3 \\ \hline
%Number of classes per package                             & 60              \\ \hline
%Number of attributes per class                            & 30               \\ \hline
%Attribute appearance probability                          & 0.5              \\ \hline
%Probability of having an attribute default value          & 0.3              \\ \hline
%Number of methods per class                               & 15               \\ \hline
%Probability of appearance for a method                    & 0.5              \\ \hline
%Number of parameters per operation                        & 4                \\ \hline
%Probability of appearance for a parameter                 & 0.5              \\  \hline
%Maximum number of generalizations                         & 300              \\ \hline
%Number of data types per package                          & 10               \\ \hline
%Probability of appearance for a structure                 & 0.3              \\ \hline
%Number of enumerations per package                        & 10               \\ \hline
%Probability of appearance for an enumeration              & 0.3              \\ \hline
%Number of function pointers per package                   & 10               \\ \hline
%Probability of appearance for a function pointer          & 0.3              \\ \hline
%\end{tabular}
%\end{table}

\subsubsection{Simulation for Law 1 right-invertibility}

In the first simulation, Law 1 is evaluated. The procedure for the simulation is shown
in Figure \ref{fig:simulation1}.

\begin{figure}
\centering
\includegraphics[width=\columnwidth]{figures/simulation1}
\caption{Simulation 1 for Law 1 (right-invertibility)} 
\label{fig:simulation1}
\end{figure}

The general idea behind the simulation procedure is to
do a full round-trip of the UML model randomly generated in step \textbf{(1)}.
The round-trip of the model is done through steps \textbf{(2)} to \textbf{(3)}.
The randomly generated model is compared with the reversed model after the round-trip. This is done in step \textbf{(5)}.

A full round-trip is also done for the C++ code that is generated
from the original randomly generated model in step \textbf{(2)}.
The round-trip is done through steps \textbf{(3)} to \textbf{(4)}. 
The code generated from the original
model is compared with code generated from the reversed model
in step \textbf{(6)}.

Code generation and reverse are done in batch mode since
no editions are made to the models or code in this simulation.
Comparison of models is done by first comparing the number of each type of elements. If these numbers match, EMF Compare is used.
Comparison of code is done by using the built-in code comparison tool in the Eclipse CDT.

Table \ref{tab:generated_models} shows the number of each type of elements in the randomly
generated model,
and the comparison results, for 3 of the 200 models created by the generator.
We limited ourselves to 200 models for practical reasons.
These same models were later used for a simulation that involved some manual editions (presented
in the next subsection).

After launching the simulation for the 200 models, no differences
were found during model comparison and code comparison.
This result assesses that the tooling of our model-code synchronization approach
with respect to Law 1 of round-trip engineering.

\begin{table*}[]
\centering
\caption{Three of 200 generated models for simulation 1: Abbreviations are classes (\tb{C}), methods (\tb{M}), method parameters (\tb{P}), attributes (A), associations (\tb{AS}), inheritances (\tb{G}), method body (\tb{B}), dependencies (\tb{D}), enumerations (\tb{E}), structures (\tb{S}), default values (\tb{DV}), function pointers (\tb{F})}
\label{tab:generated_models}
\begin{tabular}{|l|l|l|l|l|l|l|l|l|l|l|l|l|p{4pc}|p{4pc}|}
\hline
\rowcolor{Gray}
Model ID & C   & M    & P    & A    & AS   & G   & B    & D   & E  & S & DV  & F  & Differences in models? & Differences in code?\\ \hline
1        & 180 & 1419 & 2683 & 1835 & 1053 & 179 & 1138 & 507 & 13 & 7 & 150 & 11 & No & No\\ \hline
2        & 180 & 1437 & 2718 & 1874 & 1074 & 179 & 1157 & 512 & 10 & 9 & 167 & 9  & No & No\\ \hline
..       & ..  & ..   & ..   & ..   & ..   & ..   & ..  & .. & ..& ..  & ..  & .. & No & No\\ \hline
200      & 180 & 1413 & 2629 & 1857 & 1018 & 179 & 1127 & 517 & 13 & 9 & 150 & 13 & No & No\\ \hline
\end{tabular}
\end{table*}

\subsubsection{Simulation for Law 2 left-invertibility}
\label{sec:simulation2}

In the second simulation, Law 2 was evaluated. The procedure for the simulation is shown
in Figure \ref{fig:simulation2}.

\begin{figure}
\centering
\includegraphics[width=\columnwidth]{figures/simulation2}
\caption{Simulation 2 for Law 2 (left-invertibility)} 
\label{fig:simulation2}
\end{figure}

The general idea behind the simulation procedure is to
do a full round-trip of a randomly generated UML model,
but with editions introduced to the generated C++ code.
As shown in Figure \ref{fig:simulation2}, the
procedure to test our tooling consisted of the following steps:

\begin{description}
	\item[Step 1] A UML model is randomly generated and becomes the \textbf{generated model}.
	\item[Step 2] Through batch code generation, we obtain \textbf{code from generated model}.
	\item[Step 3] The code produced from generated model is then edited.
	We thus obtain \textbf{edited code}.
	The different kinds of editions will be described
	when we discuss the results of this simulation.
	\item[Step 4] The \textbf{edited code} is reversed incrementally to the \textbf{generated model}.
	The \textbf{generated model} then becomes an \textbf{updated model}.
	\item[Step 5] The \textbf{edited code} is also reversed in batch mode
	to a \textbf{model from edited code}.
	This model is an image of the edited code.
	\item[Step 6] The \textbf{updated model} (the previously \textbf{generated model})
	is compared to the \textbf{model from edited code} (image of the edited code).
	\item[Step 7] Afterwards, we also generate in batch mode \textbf{code from the updated model}.
	\item[Step 8] The \textbf{code from the updated model} is compared to the \textbf{edited code}.
\end{description}

The simulation is run for 200 randomly generated models.
We limited ourselves to 200 models because in step \textbf{(3)}, some of the editions
have to be done manually to emulate real development conditions.
Therefore the limit of 200 models is purely based on practical concerns.

During the simulations, the code generated from each model undergoes seven kinds of
editions independently (the kinds of editions performed are listed in Table \ref{tab:change_description}).
As its name suggests, only "Manual editions" were done manually by a developer, to emulate
actual development conditions.

The editions triggered three kinds of Eclipse CDT modification events:
\texttt{ADDED}, \texttt{REMOVED}, and \texttt{CHANGED}. Events \texttt{ADDED} and \texttt{REMOVED}
mean addition and deletion
of classes, attributes or methods to the code.
Event \texttt{CHANGED} is the update of elements in code including
(1) renaming attributes, classes, or methods, (2) changing the
type of attributes, parameters, or (3) changing the behavior of methods.

Table \ref{tab:change_description} shows the average number of \texttt{ADDED}, \texttt{REMOVED}, and \texttt{CHANGED} events
that were triggered by each of the seven kinds of editions. The last two columns
show the results of model comparison and code comparison for all 200 simulations following
each kind of edition (the editions are made independently).

\begin{table}[]
\centering
\caption{Editions and the modifications events they trigger (A = \texttt{ADDED}, C = \texttt{CHANGED}, R = \texttt{REMOVED})}
\label{tab:change_description}
\begin{tabular}{|p{2cm}|l|l|l|l|l|}
\hline
\rowcolor{Gray}
Edition kind                                              & A   & C    & R   & \begin{tabular}[c]{@{}l@{}}Model\\diff?\end{tabular} & \begin{tabular}[c]{@{}l@{}}Code\\diff?\end{tabular} \\ \hline
Renaming attributes of all classes                             &  0  & 1903 &  0   & \multicolumn{1}{c|}{No} & \multicolumn{1}{c|}{No} \\ \hline
Renaming methods of all classes                                &  0  & 1197 &  0   & \multicolumn{1}{c|}{No} & \multicolumn{1}{c|}{No} \\ \hline
Deleting attributes                                            &  0  &    0 & 46   & \multicolumn{1}{c|}{No} & \multicolumn{1}{c|}{No} \\ \hline
Adding attributes                                              & 25  &    0 &  0   & \multicolumn{1}{c|}{No} & \multicolumn{1}{c|}{No} \\ \hline
Adding methods                                                 & 10  &    0 &  0   & \multicolumn{1}{c|}{No} & \multicolumn{1}{c|}{No} \\ \hline
Changing method body                                           &  0  & 30   &  0   & \multicolumn{1}{c|}{No} & \multicolumn{1}{c|}{No} \\ \hline
Manual editions                                           & 39  & 34 & 30   & \multicolumn{1}{c|}{No} & \multicolumn{1}{c|}{No} \\ \hline
\end{tabular}
\end{table}

For all 200 models, no differences
were found during model comparison and code comparison.
This result assesses that the tooling of our model-code synchronization approach
with respect to Law 2 of round-trip engineering.

\subsubsection{Simulation for concurrent edition of model and code}

A third simulation aims at emulating the process described in Figure \ref{fig:scenario3} of Section \ref{sec:scenario3},
in which the model and the code are concurrently edited.
A simplified representation of the procedure of simulation 3 is shown in Figure \ref{fig:simulation3}.
The simulation is executed for 200 randomly generated
UML models and their generated C++ code.

\begin{figure}
\centering
\includegraphics[width=6.5cm]{figures/simulation3}
\caption{Simulation 3 for concurrent editions case}
\label{fig:simulation3}
\end{figure}

The idea behind the procedure in Figure \ref{fig:simulation3} is to simulate editions in the original randomly generated model,
and its corresponding generated code (step \textbf{(1)} to \textbf{(4)}).
To simplify the simulation, the simulator only introduces attribute renaming model-side.
On the code-side, the simulator only makes editions to method bodies.
Edited model and code must then be synchronized and then we must assess that they are
indeed synchronized. This is done in step \textbf{(5)}.

To synchronize edited model
and edited code, first we use the strategy based on code as the synchronization artifact. 
The simulator propagates all method body editions from the edited code to the synchronization
artifact. The simulator then propagates all attribute renaming from the synchronization
artifact to the edited code. Next, the synchronization artifact is reversed incrementally to
produce the edited model. At this point the model and the code are considered synchronized.

To assess that both synchronized model and code are images of one another, the synchronized code was
reversed in batch mode to a new model. The new model was then compared with the synchronized model. No
differences were found during these simulations. We also generated new code in batch mode
from the synchronized model. The new code was compared to the synchronized code. No differences
were found during the comparison.

The same simulation is repeated for the case where the model is used as synchronization
artifact. Again the synchronized code and model were images of one another.
%Therefore with this example of concurrent edition of model and code,
%our proposition is tested.

Once the tooling of the proposed model-code synchronization approach
was verified through the three simulations,
we decided to apply it to the development of a practical real-world system.
The results of this experiment are discussed in the next section.
