\subsection{Papyrus-RT runtime case-study}
\label{sec:case-study}

% Section introduction and goal of experiment
In Section \ref{sec:simulation} the approach was tested for UML and C++ synchronization, using simulations.
In this section we describe the application of the full synchronization approach to a case-study. The intent here was to evaluate:

\begin{itemize}
	\item The usability of the proposed synchronization approach
	\item The scalability of the tooling to a real system
\end{itemize}

% Case-study introduction
The case-study is related to the development of the runtime underlying Papyrus-RT \cite{cea_papyrus-rt_2016}.
This latter is an open-source custom modeling tool, based on Papyrus,
that supports UML extensions for the design of a category of event-driven real-time systems (UML-RT).
%UML-RT \cite{kuster_consistent_2001} offers
%a set of extensions to UML for the design of real-time systems. UML-RT is
%developed by Zeligsoft within the Papyrus-RT \cite{cea_papyrus-rt_2016} modeler dedicated to the UML-RT modeling language.
Papyrus-RT features full automated code generation for UML
models. 
The generated code executes within a corresponding runtime environment,
which provides C++ realizations of the high-level concepts defined in UML-RT.

% Motivation behind case-study
At the beginning of the project,
the resource available to develop the aforementioned runtime
was an experienced C++ programmer.
For the obvious pragmatic reasons of project management,
it was decided to implement the runtime in C++ without
the benefits of models.
However, later it was concluded that even if the result was satisfactory,
it would be beneficial to exploit the advantages that MDE provides.
Thus, an MDE approach would improve
both the maintainability and the evolvability of the runtime.
Moreover, as development progressed, the project was expanded to include
new developers, who are proponents
of MDE and are eager to work with models.

As a result, this project represented an ideal case study to assess the practicality of the proposed approach. In particular, we were interested in determining its usability and scalability

% C++ implementation and C/C++ specificities of case-study
The runtime was originally implemented as an open-source plain C++11 project.
Most of its architecture is object-oriented.
The runtime has 65 classes and 14,945 lines of code.
Other than containing typical entities found in object-oriented architectures,
the runtime uses C/C++ features such
as type definitions, templates, pointers, references, function pointers, and variadic functions to name a few.
These features are supported by the
the reverse and code generation tools in Papyrus, coupled with the C++ UML profile.

% What we did in the experiment
In order to use our model-code synchronization approach, the original Papyrus-RT runtime had first to be
reversed to a UML model. This step was crucial because the original runtime
contained some elements that were not object-oriented. Consequently,
some modifications and refactorings were made
to the original runtime to make it fully object-oriented.
This enabled it to be modeled entirely in a language like UML.
Table \ref{tab:runtime_diff} shows the two main differences between
the original runtime, and the revised object-oriented runtime.

\begin{table}
\centering
\caption{Differences in Papyrus-RT runtime versions}
\label{tab:runtime_diff}
    \begin{tabular}{|p{9pc}|p{9pc}|}
    \hline
    \rowcolor{Gray}
    Original & Object-oriented \\ \hline
    Directives to control compilation of OS-dependent code & Abstract OS-independent classes inherited by OS-dependent classes with implementation \\ \hline
    Variables, functions and type definitions outside of class & Refactored as entities inside a class. Attributes and functions are static with visibility defined according to scope of original variables and functions. \\ \hline
    \end{tabular}
\end{table}

% Results
The reverse in batch mode of the object-oriented runtime
takes about 12 seconds. All 65 classes were reversed,
with all of their attributes and methods.
Code generation in batch mode of the entire reversed UML model takes about 5 seconds
and produces 22,053 lines of code.
The difference in the number of lines of code is due to automatically
generated documentation comments.
The generated runtime compiles and the updated existing unit tests
pass when applied on the runtime compiled from generated code.

Once the runtime has been reversed,
our model-code synchronization processes could be used.
We noticed that using the proposed approach introduced MDE style development for the Papyrus-RT runtime.
This brought about several advantages as several manual tasks were automated:

\begin{itemize}
	\item Automatic handling of relationships between model elements, i.e. association, dependency, inheritance
	\item Automatic generation of includes and forward declarations in the code that avoids cyclic dependencies
	\item Graphical representation of architecture in automatically updated UML diagrams (feature of Papyrus)
\end{itemize}

In applying the proposed approach to achieve a collaborative development of the runtime system,
we found that the only real difficulty faced was initializing the synchronization processes.
This required the reverse engineering of the original runtime, with some non-object-oriented code, into an
object-oriented UML model. The reverse was successful following some modifications and refactorings.
Once the full model-code synchronization process was in place,
we were able to use it and develop concurrently both the UML model and its generated C++ code.
%The development of the runtime then benefited from automation through MDE.
