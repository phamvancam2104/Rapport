\section{Appendix}

\label{sec:example}
\begin{figure}[H]
	\centering
	\includegraphics[clip, trim=0.2cm 0.2cm 0.2cm 0.2cm, width=1.0\columnwidth]{figures/ATM.pdf}
	\caption{ATM State machine example} 
	\label{fig:example}
\end{figure}

\begin{lstlisting}[caption={IState interface and function pointers in C++, in which S0 is one of lstates and NUM\_STATES is the number of states}, label=lst:IStateCpp, language=C++]
typedef struct IState {
  int previousActives[2];   int actives[2];
  EventId defEvents[2];
  void (C::*entry)();  void (C::*exit)();  
  void (C::*doActivity)();
} IState;
class C {
private:
  IState states[NUM_STATES];
public:
  C() {
    states[S0_ID].entry = &C::S0_entry;
    defEvents[0] = EVENT_MAX; 
    defEvents[1] = EVENT_MAX;
    ...
  }
  void S0_entry {...}
}
\end{lstlisting}

\begin{lstlisting}[caption={Example code generated for doActivity, the method in takes as input a state id to use and call the appropriate mutex and \ti{doActivity}, respectively}, label=lst:doActivity, language=C++]
void doActivity(int stateId) {
  isStarts[stateId] = false;
  while(true) {
    mutex[stateId].lock();
    while(!isStarts[stateId]) {
      mutex[stateId].wait();
    }
    states[stateId].doActivity();
    isStarts[stateId] = false;
    mutex[stateId].unlock();
    if (!isStops[stateId]) {
      if (stateId == IDLE_ID || stateId == DISPENSEMONEY_ID ...) {
        pushCompletionEvent(stateId);
      }
    }
  }
}
\end{lstlisting}

\begin{lstlisting}[caption=Example code generated for the region of S1, label=lst:region, frame=single]
void S1Region1Enter(int enter_mode){
  if (enter_mode == DEFAULT) {
    states[S1_ID].actives[0] = S3_ID;
    states[S3_ID].entry();  sendStartSignal(S3_ID);
    S3Region1Enter(DEFAULT);
  } else if (enter_mode == S2_MODE) { //entry
    states[S1_ID].actives[0] = S2_ID;
    states[S2_ID].entry();  sendStartSignal(S2_ID);
  } if (enter_mode == SH_MODE) {
    StateIDEnum his;
    if (states[S1_ID].pres[0] != STATE_MAX){
      his = states[S1_ID].pres[0];
    } else {
      his = S2_ID;
    }
    states[S1_ID].actives[0] = his;
    states[his].entry();  sendStartSignal(his);
    if (S3_ID == his) {
      S3Region1Enter (S3_REGION1_DEFAULT);
    } 
  } else if (enter_mode == S4_MODE) {
    states[S1_ID].actives[0] = S3_ID;
    states[S3_ID].entry();  sendStartSignal(S3_ID);
    S3Region1Enter(S4_MODE);
  } else if (enter_mode == ENP_MODE) {...}
\end{lstlisting}

Listing \ref{lst:region} shows the C++-like example code generated for entering the state $S1$ in \ref{fig:entering}. In this case the entering model values are $\{DEFAULT = 0, SH\_MODE = 1, S2\_MODE = 2, S4\_MODE=3, ENP_MODE=4\}$. By default, the active sub-state of the region is set after the execution of any effect associated with the initial transition, $S3$ is set as active sub-state of $S1$. It is worth noting that after the execution of each entry, a start signal is sent to activate the waiting thread associated with \ti{doActivity} of the corresponding state.

\begin{lstlisting}[caption=Example code generated for $Join1$ and $Junction1$, label=lst:event1, language=C++]
if((states[VERIFYING_ID].actives[0]==CARDVALID_ID)
  &&(states[VERIFYING_ID].actives[1]==PININCORRECT_ID)) {
  Junction1 = 0; //else outgoing transition of Junction1
  if (tries < maxTries) {Junction1 = 1;}
    FORK(R1Exit); FORK(R2Exit);
    //JOIN ...
    sendStopSignal(VERIFYING_ID);	exit(VERIFYING_ID);
    FORK(effect_t11); FORK(effect_t13);
    //JOIN ...
    if (Junction1=1) {
      tries++;
      activeStateID = IDLE_ID;		entry(IDLE_ID);
      sendStartSignal(IDLE_ID);
    } else {
      cardValid = false;
      activeStateID = IDLE_ID;		sendStartSignal(IDLE_ID);
    }
  }
\end{lstlisting}